% ================================================================================
\documentclass{amsart}

%     If your article includes graphics, uncomment this command.
\usepackage{graphicx}
\usepackage{comment}
\usepackage{url}
\usepackage{hyperref}

\usepackage{listings}
\lstset{
language=Python,
basicstyle=\small\sffamily,
%basicstyle=\tiny\sffamily,
numbers=left,
numberstyle=\tiny,
frame=tb,
%frame=single,
columns=fullflexible,
showstringspaces=false
}

\newtheorem{theorem}{Theorem}[section]
\newtheorem{lemma}[theorem]{Lemma}

\theoremstyle{definition}
\newtheorem{definition}[theorem]{Definition}
\newtheorem{example}[theorem]{Example}
\newtheorem{xca}[theorem]{Exercise}

\theoremstyle{remark}
\newtheorem{remark}[theorem]{Remark}

\numberwithin{equation}{section}

%    Absolute value notation
\newcommand{\abs}[1]{\lvert#1\rvert}

%    Blank box placeholder for figures (to avoid requiring any
%    particular graphics capabilities for printing this document).
\newcommand{\blankbox}[2]{%
  \parbox{\columnwidth}{\centering
%    Set fboxsep to 0 so that the actual size of the box will match the
%    given measurements more closely.
    \setlength{\fboxsep}{0pt}%
    \fbox{\raisebox{0pt}[#2]{\hspace{#1}}}%
  }%
}
% ================================================================================
% TODO: NOTES
\begin{comment}
    \lstinputlisting[basicstyle=\tiny]{files/snowflake_tree.txt}
\end{comment}
% ================================================================================
\begin{document}
% ================================================================================
\title{Snowflake audit\\ -\\ \small{Snapshot}}
% ================================================================================
\author{Carolin Z\"obelein}
\address{Carolin Z\"obelein, Independent mathematical scientist, Josephsplatz 8, 90403 N\"urnberg, Germany, \url{https://research.carolin-zoebelein.de}}
%\curraddr{}
\email{contact@carolin-zoebelein.de, PGP: D4A7 35E8 D47F 801F 2CF6 2BA7 927A FD3C DE47 E13B}
\thanks{The author believes in the importance of the independence of research and is funded by the public community. If you also believe in this values, you can find ways for supporting the author's work here: \url{https://research.carolin-zoebelein.de/crowdfunding.html}}
% ================================================================================
%\subjclass[2000]{Primary 54C40, 14E20; Secondary 46E25, 20C20}	% TODO
\date{Last change: \today, Status: Draft}
\dedicatory{This paper is dedicated to all the brave snowflakes who die every year during winter.}
%\keywords{Differential geometry, algebraic geometry}	% TODO
% ================================================================================
%\begin{abstract} % TODO
%	STATUS: DRAFT
%\end{abstract}
% ================================================================================
\maketitle
% ================================================================================
\section*{Preamble}
\label{s:preamble}
% --------------------------------------------------------------------------------
The following document is for discussion purposses only. It gives no warranty for completeness and correctness.
% ================================================================================
\section{Introduction}
\label{s:introduction}
% --------------------------------------------------------------------------------
\textit{Snowflake} is a pluggable transport, which uses WebRTC to proxy traffic through emporary proxies. It aims to work kind of like flash proxy \cite{SnowflakeEmail} \cite{SnowflakeGitWeb}. This document is a short snapshot audit of the current state of the existing snowflake code  \cite{SnowflakeGit} on \url{https://gitweb.torproject.org/pluggable-transports/snowflake.git/}.
% ================================================================================
\section{Basic package structure}
\label{s:basicpackagestructure}
% --------------------------------------------------------------------------------
At first, let's have a first look at the basic package structure of Snowflake \cite{SnowflakeGit}.
% --------------------------------------------------------------------------------
\subsection{Package tree}
\label{ss:packagetree}
% --------------------------------------------------------------------------------
Given are the following main parts of the package:
\begin{itemize}
    \item \textbf{appengine (d):} Runs an Google App Engine and reflects domain-fronted requests from a client to the Snowflake broker.
    \item \textbf{broker (d):} Handles the rendezvous by matching Snowflake clients with proxies. It passes the client's WebRTC session descriptions. So the clients and proxies can establish a peer connection.
    \item \textbf{client (d):} Tor client component of Snowflake.
    \item \textbf{proxy (d):} Browser proxy component of Snowflake.
    \item \textbf{proxy-go (d):} Standalone version of the Snowflake proxy.
    \item \textbf{server (d):} Server transport plugin for Snowflake. The client connects to the proxy using WebRTC and the proxy connects to the server using WebSocket.
    \item \textbf{server-webrtc (d):} WebRTC server plugin which uses an HTTP server that simulates the interaction that a client would have with the broker, for direct testing.
    \item \textbf{CONTRIBUTING.md (f), LICENSE (f), README.md (f):} Standard package files.
\end{itemize}

d= directory, f=file.
% ================================================================================
\section{Conclusion}
\label{s:conclusion}
% --------------------------------------------------------------------------------
% TODO
% ================================================================================
% Bibliography
\nocite{*}
%\bibliographystyle{amsplain}
\bibliographystyle{unsrtdin}
\bibliography{snowflake-audit}
% ================================================================================
\section*{License}
\label{s:license}
% --------------------------------------------------------------------------------
\begin{center}
	\includegraphics{by-nc-nd.png} \\
	\url{https://creativecommons.org/licenses/by-nc-nd/4.0/}
\end{center}
% ================================================================================
\end{document}
% ================================================================================
